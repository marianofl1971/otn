\documentclass[spanish,11pt,oneside,a4paper]{book}
% El siguiente paquete se utiliza para la división en sílabas en español. IMPORTANTE: requiere instalación de paquetes texlive-babel-spanish y texlive-babel-spanish-doc (ambos sin t al final de tex). La opción es-noshorthands es para evitar el problema que se ocasiona cuando la opción spanish deshabilita algunos caracteres.
\usepackage[spanish,es-noshorthands]{babel}
\usepackage[utf8x]{inputenc}
\usepackage[T1]{fontenc}
\usepackage{cite}
\usepackage{graphicx}
\usepackage{longtable}
\usepackage{enumitem}
\usepackage{verbatim}
\usepackage{verbatim}
\usepackage{hyperref}
\usepackage{listings}
\usepackage{eurosym} % Para poder escribir el símbolo del euro
\lstset{escapeinside=||} % Para poder poner acentos en dentro del entorno lstlisting. Por ejemplo, para poner "Fernández" en unos comentarios de Java, hay que escribir "Fern|á|ndez"
\usepackage{framed} % Para poder crear cuadros alrededor de textos. Permite utilizar el entorno framed.
\usepackage{courier}
\lstset{basicstyle=\footnotesize\ttfamily,breaklines=true}
\usepackage{tikz} % Para las figuras en formato LaTeX
\usetikzlibrary{arrows,positioning,calc,shapes.geometric}
\usepackage{amsthm} % Este paquete lo utilizo para crear entornos ejemplo, ejercicio, etc. con newtheorem.
\usepackage{amsmath} % Lo utilizo para poder utilizar el entorno equation*, que permite tener fórmulas si número de fórmula.
\usepackage{float} % Para poder poner las figuras donde yo quiera.
\usepackage{amsfonts} % Para poder escribir el símbolo de los números naturales, el de los números enteros, etc.
\usepackage{booktabs} % Para poder utilizar \toprule, midrule, etc. en tablas.
\usepackage{tikz} % Para poder dibujar, por ejemplo, aut�matas
\usetikzlibrary{automata,positioning} % Para poder dibujar aut�matas


%\usepackage{array}

%\input{latex-macros}

\begin{document}
\theoremstyle{definition}
\newtheorem{ejercicio}{Ejercicio}[chapter] %El primer parámetro es el nombre del entorno y, el segundo, el nombre que va a aparecer en el texto. Lo que hay entre corchetes indica que la numeración va a llevar el número de sección.
\newtheorem{definimos}{Definición}[chapter] 
\newtheorem{teorema}{Teorema}[chapter] 
\newtheorem{observamos}{Observación}[chapter] 
\newtheorem{ejemplo}{Ejemplo}[chapter] 
%\renewcommand{\listtablename}{�ndice de tablas}
\renewcommand{\tablename}{Tabla}
\begin{titlepage}

\topskip0pt
\vspace*{\fill}
\begin{center}

    {\huge \textbf{Ontolgía de Planes de Estudio (OPE)}}
    \\{\LARGE Especificación de requisitos}
\vspace{3cm}

{\Large Mariano Fern\'andez L\'opez\\}
{\Large Escuela Polit\'ecnica Superior, Universidad San Pablo CEU\\}

{\Large \today}
\end{center}

\vspace*{\fill}

\end{titlepage}
\setcounter{page}{1}

\tableofcontents

Blablabla said Nobody ~\cite{Nobody06}.

\bibliography{mybib}{}
\bibliographystyle{plain}

\end{document}
