% FILE NAME: contenido.tex

\sloppy % Para que no se me salgan las URLs de los m\'argenes

\chapter{Información general}

\section{Propósito}

El propósito de la elaboración de esta ontología es especificar de manera
formal una estructura conceptual para el intercambio de información de acuerdo
con la Guía Práctica para la Evaluación del Seguimiento de Títulos
Universitarios Oficiales de la Fundación Madri+d\footnote{\url{http://www.madrimasd.org/uploads/acreditacion/doc/seguimiento2013/Guia\_de\_Evaluacion\_2014.pdf}}. La ontología estará centrada en la información de los títulos que, según Madri+d, debe ser pública.

Se espera que esta ontología sea útil para desarrollar sistemas de ayuda a la
decisión, orientados a futuros estudiantes universitarios, que traten de forma
automatizada los diferentes títulos sin necesidad de programar un
\emph{scrapper} para el sitio Web de cada título. También puede resultar útil
para intercambio de información entre las aplicaciones de Madri+d y las de las
universidades.

\section{Recursos para elaborar la ontología}

La identificación de las entidades que se han de definir en la ontología se llevará a cabo tomando como referencia el documento de Madri+d anteriormente citado.

Las definiciones de los términos que aparecen en la ontología se obtendrán a partir de la Ley Orgánica 6/2001, de 21 de diciembre, de Universidades, en su versión consolidada de 23 de marzo de 2016\footnote{\url{http://www.boe.es/buscar/act.php?id=BOE-A-2001-24515}}.

\chapter{Descripción de la ontologia en lenguaje natural}

Los apartados siguientes están estructurados según el capítulo 2 del documento de la Fundación Madri+d anteriormente mencionado (sobre la información pública). En cada uno de ellos se expondrá qué información deberá estar representada en la ontología.

%Los distintos puntos estarán clasificados según su prioridad. Los que tengan un número de prioridad menor serán más prioritarios.

\section{Descripción del título}

Blablabla said Nobody ~\cite{Nobody06}.

\begin{enumerate}

    \item Denominación del título.

    \item Universidad donde se imparte.

    \item Centro, Departamento o Instituto responsable

    \item Centros en los que se imparte el título.

    \item Curso académico en el que se implantó.

    \item Tipo de enseñanza (presencial, semi-presencial, a distancia).

    \item Número total de créditos ECTS.

    \item Número mínimo y máximo de ECTS por tipo de matrícula y curso. Así, tanto para
        los estudiantes a tiempo completo, como para los estudiantes a tiempo
        parcial, se representará el número mínimo y máximo de ECTS de los que
        podrá matricularse durante el primer curso, y la misma información para
        el resto de cursos. Estos máximos y mínimos serán los mismos para
        segundo curso, tercer curso, etc.

    \item Recurso donde se describe la normativa de permanencia.

    \item Idiomas en que se imparte el título.

\end{enumerate}

\section{Competencias}

Por competencia se entiende ``el conjunto de conocimientos, habilidades,
actitudes que se adquieren o desarrollan mediante experiencias formativas
coordinadas, las cuales tienen el propósito de lograr conocimientos
funcionales que den respuesta de modo eficiente a una tarea o problema de la
vida cotidiana y profesional que requiera un proceso de enseñanza y
aprendizaje''%\cite{madriMasDGuiaSeg}.

\begin{enumerate}

    \item Competencias generales.

    \item Competencias transversales.

    \item Competencias específicas.

    \item Profesiones reguladas para las que capacita el título, en su caso.

\end{enumerate}

\section{Acceso y admisión}

\begin{enumerate}

    \item{Información dirigida al estudiante de nuevo ingreso}

        \begin{enumerate}

            \item Recurso donde se describen las vías y requisitos de acceso (GRADOS).

            \item Recurso donde se describen los criterios de admisión (MASTERS).

            \item Número de plazas de nuevo ingreso ofertadas.

            \item Recurso donde se describen las pruebas especiales, en su caso.

            \item Plazos de preinscripción.

            \item Periodo y recurso donde se describen los requisitos para formalizar
                la matrícula.

            \item Recurso donde se describe el perfil recomendado para el
                estudiante de nuevo ingreso.

        \end{enumerate}

    \item Recurso con la información sobre transferencia y reconocimiento de créditos.

    \item Recurso con el procedimiento de adaptación de los estudiantes procedentes de enseñanzas anteriores (sólo en el caso de que el título provenga de la transformación a la nueva legislación de otro título).

    \item Cursos de adaptación (plan curricular y condiciones de acceso).

    \item Recurso donde se describen los mecanismos de información y orientación para estudiantes matriculados.

\end{enumerate}

\section{Planificación de las enseñanzas}

\begin{enumerate}

    \item Calendario de implantación del código.

    \item Información con secuencia temporal ( curso/semestre) en la que se haga
        constar el nombre de la materia, su carácter, el número de créditos ECTS
        y las asignaturas en las que se desdoble dicha materia con indicación del
        número de créditos ECTS asignado a cada una de ellas.

    \item Competencias asociadas a cada una de las materias.

    \item Itinerarios formativos (menciones / grados – especialidades/ másteres).

    \item Guías docentes de las asignaturas:

        \begin{itemize}

            \item Tipo de asignatura.
            \item Número de créditos.
            \item Programa.
            \item Objetivos de aprendizaje.
            \item Metodología de aprendizaje.
            \item Criterios de evaluación.
            \item Idioma.

        \end{itemize}

    \item Acuerdos o convenios de colaboración y programas de ayuda para el
        intercambio de estudiantes

    \item Prácticas externas (convenios con entidades públicas o privadas, sistema
        de tutorías, sistemas de solicitud, criterios de adjudicación, etc.).


\end{enumerate}

\section{Personal académico}

\begin{enumerate}

    \item Personal académico adscrito al título.

    \item Para las universidades públicas, se requiere la siguiente información agregada:

        \begin{itemize}

            \item Número total de profesores por categoría.

            \item Porcentaje de doctores por categoría.

        \end{itemize}

    \item La información agregada para las universidades privadas, será la que
        se muestra a continuación:

        \begin{itemize}

            \item Número total de profesores.

            \item Porcentaje de doctores.

        \end{itemize}

\end{enumerate}

\section{Medios materiales}

 Aulas informáticas, recursos bibliográficos, bibliotecas, salas de estudio.

\section{Sistema interno de calidad}

\begin{enumerate}

    \item Breve descripción de la organización ,composición y funciones del SICG
        (título o centro)

    \item Mejoras implantadas.

    \item Información sobre acceso al sistema de quejas y reclamaciones.

    \item Información sobre la inserción laboral de los graduados.

\end{enumerate}

\bibliography{bibText}{}
\bibliographystyle{plain}
