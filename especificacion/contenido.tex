% FILE NAME: contenido.tex

\sloppy % Para que no se me salgan las URLs de los m\'argenes

\chapter{Información general}

\section{Propósito}

El propósito de la elaboración de esta ontología es especificar de manera
formal una estructura conceptual para el intercambio de información de acuerdo
con la Guía Práctica para la Evaluación del Seguimiento de Títulos
Universitarios Oficiales de la Fundación Madri+d\cite{guiaMadridMasD}.
La ontología estará centrada en la información de los títulos oficiales que,
según Madri+d, debe ser pública. Esta guía es aplicable a las 14
universidades de la Comunidad de
Madrid\footnote{\url{http://www.emes.es/Sistemauniversitario/UniversidadesdeMadrid/tabid/215/Default.aspx}}.

Se espera que esta ontología sea útil para desarrollar sistemas de ayuda a la
decisión, orientados a futuros estudiantes universitarios, que traten de forma
automatizada los diferentes títulos sin necesidad de programar un
\emph{scrapper} para el sitio Web de cada título. También puede resultar útil
para intercambio de información entre las aplicaciones de Madri+d y las de las
universidades.

\section{Recursos para elaborar la ontología}

La identificación de las entidades que se han de definir en la ontología se
llevará a cabo tomando como referencia el documento de Madri+d anteriormente
citado\cite{guiaMadridMasD}.

Las definiciones de los términos que aparecen en la ontología se obtendrán a
partir de la siguiente legislación:

\begin{enumerate}

    %    \item Real  Decreto  898/1985,  de  30  de  abril,  sobre  régimen  del 
    %    profesorado universitario\cite{regimenProfesorado}.

    \item Real Decreto 557/1991, de 12 de abril, sobre creación y
        reconocimiento de universidades y centros universitarios
        \cite{rdUniversidadesPrivadas}.

    \item Ley Orgánica de Universidades\cite{leyUniversidades}.

    \item Real Decreto 1125/2003, de 5 de septiembre, por el que se establece
        el sistema europeo de créditos y el sistema de calificaciones en las
        titulaciones universitarias de carácter oficial y validez en todo el
        territorio nacional\cite{rdECTS}.

    \item Real Decreto 1393/2007, de 29 de octubre, por el que se establece
        la ordenación de las enseñanzas universitarias oficiales\cite{rd1393}
        (en aquello en lo que no haya sido modificado por el Real Decreto
        861\cite{rd861}).

    \item Real Decreto 861/2010, de 2 de julio, por el que se modifica el Real
        Decreto 1393/2007, de 29 de octubre, por el que se establece la
        ordenación de las enseñanzas universitarias oficiales\cite{rd861}.

    \item Estatuto del Estudiante Universitario\cite{estatutoEstudiante}.

    \item Guía de apoyo para la elaboración de la memoria de verificación
        de títulos oficialaes universitarios\cite{guiaAneca}.

    \item En caso de que una definición no se encuentre en ninguna de las
        fuentes anteriores, se buscará en el diccionario de la Real Academia
        Española (RAE)\footnote{\url{http://dle.rae.es}}.

    \item Si ninguna de las fuentes es adecuada para alguna de las
        definiciones, entonces será proporcionada por el autor del documento,
        que es profesor y coordinador de titulación.

\end{enumerate}

\chapter{Sobre el contenido de la ontologia}

Los apartados siguientes están estructurados según el capítulo 2 del documento de la Fundación Madri+d\cite{guiaMadridMasD} (sobre la información pública). En cada uno de ellos se expondrá qué información deberá estar representada en la ontología.

%Los distintos puntos estarán clasificados según su prioridad. Los que tengan un número de prioridad menor serán más prioritarios.

\section{Descripción del título}

\begin{enumerate}

    \item Denominación del título.

    \item Universidad donde se imparte.

    \item Centro, Departamento o Instituto responsable

    \item Centros en los que se imparte el título.

    \item Curso académico en el que se implantó.

    \item Tipo de enseñanza (presencial, semi-presencial, a distancia).

    \item Número total de créditos ECTS.

    \item Número mínimo y máximo de ECTS por tipo de matrícula y curso. Así, tanto para
        los estudiantes a tiempo completo, como para los estudiantes a tiempo
        parcial, se representará el número mínimo y máximo de ECTS de los que
        podrá matricularse durante el primer curso, y la misma información para
        el resto de cursos. Estos máximos y mínimos serán los mismos para
        segundo curso, tercer curso, etc.

    \item Recurso donde se describe la normativa de permanencia.

    \item Idiomas en que se imparte el título.

\end{enumerate}

\section{Competencias}\label{competencias}

Por competencia se entiende ``el conjunto de conocimientos, habilidades,
actitudes que se adquieren o desarrollan mediante experiencias formativas
coordinadas, las cuales tienen el propósito de lograr conocimientos
funcionales que den respuesta de modo eficiente a una tarea o problema de la
vida cotidiana y profesional que requiera un proceso de enseñanza y
aprendizaje''\cite{guiaMadridMasD} (página 13).

\begin{enumerate}
        %rd861
    \item \emph{Competencias básicas o generales}, que son comunes a la mayoría de los
        títulos pero están adaptadas al contexto específico de cada uno de ellos.
        Dentro de este bloque se pueden encontrar competencias personales, competencias
        interpersonales, etc.

    \item \emph{Competencias específicas}, que son propias de un ámbito o título
        y están orientadas a la consecución de un perfil específico de
        egresado. En general, acostumbran a tener una proyección
        longitudinal en el título.

    \item \emph{Competencias transversales}, que son comunes a todos los
        estudiantes de una misma Universidad o centro universitario,
        independientemente del título que cursen.

    \item \emph{Profesiones reguladas} para las que capacita el
        título, en su caso.

\end{enumerate}

\section{Acceso y admisión}

\begin{enumerate}

    \item{Información dirigida al estudiante de nuevo ingreso}

        \begin{enumerate}

            \item Recurso donde se describen las vías y requisitos de acceso (GRADOS).

            \item Recurso donde se describen los criterios de admisión (MASTERS).

            \item Número de plazas de nuevo ingreso ofertadas.

            \item Recurso donde se describen las pruebas especiales, en su caso.

            \item Plazos de preinscripción.

            \item Periodo y recurso donde se describen los requisitos para formalizar
                la matrícula.

            \item Recurso donde se describe el perfil recomendado para el
                estudiante de nuevo ingreso.

        \end{enumerate}

    \item Recurso con la información sobre transferencia y reconocimiento de créditos.

    \item Recurso con el procedimiento de adaptación de los estudiantes procedentes de enseñanzas anteriores (sólo en el caso de que el título provenga de la transformación a la nueva legislación de otro título).

    \item Cursos de adaptación (plan curricular y condiciones de acceso).

    \item Recurso donde se describen los mecanismos de información y orientación para estudiantes matriculados.

\end{enumerate}

\section{Planificación de las enseñanzas}

\begin{enumerate}

    \item Calendario de implantación del título.

    \item Información con secuencia temporal (curso/semestre) en la que se haga
        constar el nombre de la materia, su carácter, el número de créditos ECTS
        y las asignaturas en las que se desdoble dicha materia con indicación del
        número de créditos ECTS asignado a cada una de ellas.

    \item Competencias asociadas a cada una de las materias.

    \item Itinerarios formativos (menciones / grados – especialidades/ másteres).

    \item Guías docentes de las asignaturas. Cada una de ellas deberá contener
        la siguiente información:

        \begin{itemize}

            \item Tipo de asignatura (carácter de la asignatura: básica, obligatoria u optativa).
            \item Número de créditos.
            \item Programa.
            \item Objetivos de aprendizaje.
            \item Metodología de aprendizaje.
            \item Criterios de evaluación.
            \item Idioma.

        \end{itemize}

    \item Acuerdos o convenios de colaboración y programas de ayuda para el
        intercambio de estudiantes

    \item Prácticas externas (convenios con entidades públicas o privadas, sistema
        de tutorías, sistemas de solicitud, criterios de adjudicación, etc.).


\end{enumerate}

\section{Personal académico}

\begin{enumerate}

    \item Personal académico adscrito al título.

    \item Para las universidades públicas, se requiere la siguiente información agregada:

        \begin{itemize}

            \item Número total de profesores por categoría.

            \item Porcentaje de doctores por categoría.

        \end{itemize}

    \item La información agregada para las universidades privadas, será la que
        se muestra a continuación:

        \begin{itemize}

            \item Número total de profesores.

            \item Porcentaje de doctores.

        \end{itemize}

\end{enumerate}

\section{Medios materiales}

Aulas informáticas, recursos bibliográficos, bibliotecas, salas de estudio.

\section{Sistema interno de calidad}

\begin{enumerate}

    \item Recurso con una breve descripción de la organización ,composición y funciones del SICG
        (título o centro)

    \item Recurso con una breve descripción de las mejoras implantadas.

    \item Recurso con la información sobre acceso al sistema de quejas y reclamaciones.

    \item Recurso con la información sobre la inserción laboral de los graduados.

\end{enumerate}

\chapter{Glosario de términos}

A continuación se enumeran los términos que referencian a entidades del dominio:

\begin{enumerate}

    \item \emph{título oficial} \label{def_titulo_documento} Documento expedido
        en nombre del Rey por el Rector de una universidad \cite[artículo
        35]{leyUniversidades} que acredita una serie de conocimientos
        científicos, técnicos o artísticos \cite[artículo
        33]{leyUniversidades}.

    \item \emph{plan de estudios}. Programa en el que se detallan las
        asignaturas que hay que cursar para obtener un título, así como los
        medios, tanto humanos como materiales, que se van a emplear para
        llevara buen término el programa.

    \item \emph{universidad}. Institución de educación superior dedicada a la
        investigación, la docencia y el estudio \cite[articulo 1, apartado 
        1]{leyUniversidades}.

    \item \emph{universidad pública}. Universidad cuya titularidad ostenta el Estado o
        una Comunidad Autónoma \cite[artículo 3]{rdUniversidadesPrivadas}.

    \item \emph{universidad privada}. Universidad cuya titularidad ostenta una persona
        física o jurídica de carácter privado\cite[artículo
        3]{rdUniversidadesPrivadas}.

    \item \emph{centro}. Se asumirán dos acepciones:

        \begin{enumerate}

            \item \label{centro_div} División académica de una universidad, en la que se agrupan
                los estudios de disciplinas similares (adaptado de la
                RAE\footnote{\url{http://dle.rae.es/?id=HTxyZDZ}}). Un centro
                está encargado de la organización de las enseñanzas y de los
                procesos académicos, administrativos y de gestión conducentes a
                la obtención de títulos universitarios \cite[artículo 8]{leyUniversidades}.

            \item Local o conjunto de locales en que funciona un centro según la acepción \ref{centro_div} (adaptado de la
                RAE\footnote{\url{http://dle.rae.es/?id=HTxyZDZ}}).
        \end{enumerate}

    \item \emph{departamento}. unidad de docencia e investigación encargada de
        coordinar las enseñanzas de uno o varios ámbitos del conocimiento en uno o varios centros,
        de acuerdo con la programación docente de la universidad, de apoyar las actividades e
        iniciativas docentes e investigadoras del profesorado, y de ejercer aquellas otras funciones
        que sean determinadas por los estatutos\cite[artículo 9]{leyUniversidades}.


        %    \item instituto.

    \item \emph{curso académico}. Intervalo de tiempo anual en el que se desarrolla la actividad académica. 

    \item \emph{enseñanza presencial}. Aquélla que requiere que el estudiante
        asista de forma regular y continuada durante todos los cursos a
        actividades formativas regladas en el centro de impartición del título\cite[página 10]{guiaMadridMasD}.

    \item \emph{enseñanza semi-presencial}. Aquélla en la que la planificación de las
        actividades formativas previstas en el Plan de Estudios combina la
        presencia física del estudiante en el centro de impartición del título con un
        mayor trabajo autónomo del estudiante al propio de la enseñanza
        presencial\cite[página 10]{guiaMadridMasD}.

    \item \emph{enseñanza a distancia}. Aquélla en que la gran mayoría de las actividades
        formativas previstas en el Plan de Estudios no requieren la presencia física
        del estudiante en el centro de impartición del título\cite[página 10]{guiaMadridMasD}.

    \item \emph{crédito ECTS}. El crédito europeo es la unidad de medida del
        haber académico que representa la cantidad de trabajo del estudiante
        para cumplir los objetivos del programa de estudios y que se obtiene
        por la superación de cada una de las materias que integran los planes
        de estudios de las diversas enseñanzas conducentes a la obtención de
        títulos universitarios de carácter oficial y validez en todo el
        territorio nacional. En esta unidad de medida se integran las
        enseñanzas teóricas y prácticas, así como otras actividades académicas
        dirigidas, con inclusión de las horas de estudio y de trabajo que el
        estudiante debe realizar para alcanzar los objetivos formativos propios
        de cada una de las materias del correspondiente plan de
        estudios\cite[artículo 3]{rdECTS}. El número mínimo de horas, por
        crédito, será de 25, y el número máximo, de 30\cite[artículo 4, apartado 
        5]{rdECTS}.

    \item \emph{estudiante}. Toda  persona  que  curse  enseñanzas  oficiales
        en alguno de los tres ciclos universitarios (Grado, Máster y Doctorado
        \cite[artículo 8]{rd1393}), enseñanzas de formación continua u otros
        estudios ofrecidos por las universidades\cite[artículo
        1, apartado 3]{estatutoEstudiante}.

    \item \emph{estudiante a tiempo completo}. Estudiante que dedica un mínimo
        de 36 horas y un máximo de 40 a la semana a sus
        universitarios\cite[artículo 4, apartado 4]{rdECTS}.

    \item \emph{estudiante a tiempo parcial}. Estudiante que requiere una
        trayectoria de aprendizaje flexible \cite[artículo 7, apartado 
        2]{estatutoEstudiante}, y que, por tanto, no llega a una dedicación de
        36 horas semanales a sus estudios universitarios. 

    \item \emph{normativa de permanencia}. Conjunto de normas que establecen
        bajo qué condiciones un estudiante puede permanecer en la universidad.

    \item \label{idioma} \emph{idioma}. Lengua de un pueblo o nación, o común a
        varios\footnote{\url{http://dle.rae.es/?id=KuMp7nw} acepción 1}.

    \item \emph{competencia}. Véase la sección \ref{competencias}.
        En el ámbito universitario, las competencias deben ser
        evaluables y coherentes \cite[página 19]{guiaAneca}.

    \item \emph{competencia básica}. Véase la sección \ref{competencias}.

    \item \emph{competencia específica}. Véase la sección \ref{competencias}.

    \item \emph{competencia transversal}. Véase la sección \ref{competencias}.

    \item \emph{profesión}. Empleo, facultad u oficio que alguien ejerce y por
        el que percibe una
        retribución\footnote{\url{http://dle.rae.es/?id=UHx86MW} acepción 2}.

    \item \emph{profesión regulada por exigencia de título universitario}:
        según \cite[artículo 4, apartado c]{rdECTS}, aquella profesión para cuyo
        acceso se exija estar en posesión de un título universitario oficial
        cuyo diseño y directrices respondan a lo dispuesto en los artículos
        12.9 y 15.4 del Real Decreto 1393/2007\cite{rd1393}.

    \item \emph{descripción en lenguaje natural}. Descripción utilizando un idioma humano (véase \ref{idioma}). 

    \item \emph{número de plazas}. Número de nuevos estudiantes que pueden
        admitirse a lo largo de un curso académico en una titulación.

    \item \emph{plazo de preinscripción}. Intervalo de tiempo en el que se
        puede solicitar el ingreso en una tituación.

    \item \emph{calendario de implantación del título}. Secuencia temporal en la
        que se van implantando los diferentes cursos de la
        titulación\cite[página 62]{guiaAneca}.

    \item \emph{rama de conocimiento}. Área del saber que, según el RD 1393
        \cite[artículo 12, apartado 4]{rd1393}, puede ser una de las siguientes:
        Artes y Humanidades, Ciencias, Ciencias de la salud, Ciencias Sociales
        y Jurídicas, e Ingeniería y Arquitectura. Todo título debe estar
        adscrito a, al menos, una rama de conocimiento.

    \item \emph{materia}. Unidad académica que incluye una o varias asignaturas que
        pueden concebirse de manera integrada, de tal forma que constituyen
        unidades coherentes desde el punto de vista disciplinar \cite[página
        33]{guiaAneca}.

    \item \emph{módulo}. Unidad académica que incluye una o varias materias que
        constituyen una unidad organizativa dentro de un plan de estudios
        \cite[página 33]{guiaAneca}.

    \item \emph{asignatura}. Unidad académica que incluye una serie de
        contenidos coherentes desde el punto de vista disciplinar de la que es
        evaluado el estudiante y se le asigna una calificación.

    \item \emph{carácter de materia/asignatura}. Clasificación de la materia/asignatura en
        básica, obligatoria, optativa, seminario, prácticas externas,
        trabajo dirigido, trabajo fin de grado, trabajo de fin de máster, o
        mixta.\cite[artículo 12, apartado 2]{rd1393} \cite[sección
        D.5]{guiaMadridMasD}

    \item \emph{materia/asignatura básica}. Materia/asignatura en que el estudiante adquiere las
        competencias fundamentales de la rama de conocimiento \cite[artículo
        12, apartado 2]{rd1393}. El RD 1393 \cite[anexo II]{rd1393} proporciona
        un catálogo de materias básicas por ramas de conocimiento. Son las que
        se muestran a continuación:

        \begin{itemize}

            \item Artes y Humanidades

                \begin{itemize}

                    \item Antropología.
                    \item Arte.
                    \item Ética.
                    \item Expresión Artística.
                    \item Filosofía.
                    \item Geografía.
                    \item Historia.
                    \item Idioma Moderno.
                    \item Lengua.
                    \item Lengua Clásica.
                    \item Lingüística.
                    \item Literatura.
                    \item Sociología.

                \end{itemize}

            \item Ciencias
                \begin{itemize}

                    \item Biología.

                    \item Física.

                    \item Geología.

                    \item Matemáticas.

                    \item Química.

                \end{itemize}

            \item Ciencias de la Salud.

                \begin{itemize}
                    \item Anatomía Animal.
                    \item Anatomía Humana.
                    \item Biología.
                    \item Bioquímica.
                    \item Estadística.
                    \item Física.
                    \item Fisiología.
                    \item Psicología.
                \end{itemize}

            \item Ciencias Sociales y Jurídicas.
                \begin{itemize}

                    \item Antropología.
                    \item Ciencia Política.
                    \item Comunicación.
                    \item Derecho.
                    \item Economía.
                    \item Educación.
                    \item Empresa.
                    \item Estadística.
                    \item Geografía.
                    \item Historia.
                    \item Psicología.
                    \item Sociología.
                \end{itemize}

            \item Ingeniería y Arquitectura.
                \begin{itemize}

                    \item Empresa.
                    \item Expresión Gráfica.
                    \item Física.
                    \item Informática.
                    \item Matemáticas.
                    \item Química.

                \end{itemize}

        \end{itemize}

        El plan de estudios de cualquier titulación deberá contener un mínimo de
        60 créditos de formación básica, de los que, al menos, 36
        estarán vinculados a algunas de las materias del listado anterior
        para la rama de conociento a la que esté adscrito el título\cite[artículo 12,
        apartado 5]{rd1393}.

        Estas materias deberán concretarse en asignaturas con un mínimo
        de 6 créditos cada una y serán ofertadas en la primera
        mitad del plan de estudios.

        Los créditos restantes hasta 60, en su caso, deberán estar configurados por
        materias básicas de la misma u otras ramas de conocimiento de las incluidas en
        el listado anterior, o por otras materias siempre que se justifique su carácter
        básico para la formación inicial del estudiante o su carácter transversal.

    \item \emph{materia/asignatura obligatoria}. Aquélla no básica que debe ser cursada
        necesariamente para obtener el título\cite[sección
        D.5]{guiaMadridMasD}\cite[artículo 12, apartado 2]{rd1393}.

    \item \emph{materia/asignatura optativa}. Aquélla que elige el estudiante entre
        otras posibles para cursarla\cite[sección
        D.5]{guiaMadridMasD}\cite[artículo 12, apartado 2]{rd1393}.

    \item \emph{materia mixta}. Materia que engloba asignaturas de diferente
        carácter\cite[sección D.5]{guiaMadridMasD}\cite[artículo 12, apartado 
        2]{rd1393}. 

    \item \emph{itinerario formativo}. Mención (para el caso de un grado),
        especialidad (para el caso de un máster) \cite[sección
        D.4]{guiaMadridMasD}\cite[artículo 9, apartado 3]{rd1393}.

    \item \emph{trabajo fin de grado}. Trabajo con el que concluye un título de
        grado\cite[artículo 12, apartado 3]{rd1393}. El trabajo de fin de Grado
        tendrá entre 6 y 30 créditos, deberá realizarse en la fase final del
        plan de estudios y estar orientado a la evaluación de competencias
        asociadas al título.\cite[artículo 12, apartado 7]{rd1393}

    \item \emph{trabajo fin de máster}. Trabajo con el que concluye un título
        de máster\cite[artículo 15, apartado 3]{rd1393}. Tendrá entre 6 y 30 créditos.

    \item \emph{guía docente}. Especificación de una asignatura que incluye su
        número de créditos ECTS, el programa a desarrollar, la metodología de
        aprendizaje, los sistemas de evaluación, etc. \cite[sección
        D.5]{guiaMadridMasD}.

    \item \emph{profesor}. Persona que se dedica a la docencia en la
        universidad.

    \item \emph{profesor ayudante}. Profesor en periodo de formación, con
        contrato de carácter temporal y dedicación a tiempo completo, que ha
        sido admitido, o está en condiciones de ser admitido, en estudios de
        doctorado\cite[artículo 49]{leyUniversidades}. 

    \item \emph{profesor ayudante doctor}. Profesor doctor, con contrato de
        carácter temporal y dedicación a tiempo completo, que ha obtenido una
        acreditación nacional o autonómica, y que desarrolla tareas docentes y de
        investigación\cite[artículo 50]{leyUniversidades}.  

    \item \emph{profesor titular de universidad}. Profesor funcionario doctor con plena
        capacidad tanto para la docencia como para la
        investigación. Para acceder a esta posición es necesario haber obtenido
        una acreditación nacional\cite{leyUniversidades}.

    \item \emph{catedrático de universidad}. Profesor funcionario doctor con plena
        capacidad tanto para la docencia como para la investigación que tiene
        la categoría docente más alta. Para acceder a esta posición es
        necesario haber obtenido una acreditación nacional\cite{leyUniversidades}.

    \item \emph{profesor contratado doctor}. Profesor doctor con contrato
        laboral indefinido y dedicación a tiempo completo que ha obtenido una
        acreditación nacional o autonómica, que desarrolla, con plena capacidad
        docente e investigadora, tareas docentes y de investigación (o
        prioritariamente de investigación)\cite[artículo 52]{leyUniversidades}.

    \item \emph{profesor asociado}. Profesor, con contrado temporal a tiempo
        parcial, especialista de reconocida competencia que acredita ejercer su
        actividad profesional fuera del ámbito académico universitario que
        desarrolla tareas docentes a través de las que aporta sus conocimientos
        y experiencia profesionales a la universidad\cite[artículo 53]{leyUniversidades}.

    \item \emph{profesor visitante}. Profesor o investigador, con contrato
        temporal a tiempo parcial o completo, de reconocido prestigio de otra
        universidad o centro de investigación, que desarrolla tareas docentes o
        de investigación a través de las que aporta conocimientos y experiencia
        docente e investigadora\cite[artículo 54]{leyUniversidades}.

    \item \emph{profesor emérito} Profesor jubilado que ha que ha prestado
        servicios destacados a la universidad\cite[artículo 54bis]{leyUniversidades}.

    \item \emph{doctor}. Persona que ha defendido con éxito una tesis doctoral,
        consistente en un trabajo original de investigación y que, por tanto,
        posee el título de doctor.

    \item \emph{aula}. Sala donde se imparten clases en los centros
        docentes\footnote{\url{http://dle.rae.es/?id=4OCO4gi} acepción 1}.

    \item \emph{aula informática}. Aula con ordenadores para que puedan
        trabajar con ellos los alumnos.

    \item \emph{recurso bibliográfico}. Libro, revista o cualquier otra obra
        científica o literaria que puede estar impresa o en otro soporte. 

    \item \emph{biblioteca}. Órgano cuya finalidad consiste en la adquisición,
        conservación, estudio y exposición de libros y
        documentos\footnote{Inspirado en \url{http://dle.rae.es/?id=5SGETnQ}
        acepción 1}.

    \item \emph{sala de estudio}. Sala debidamente acondicionada para que los
        alumnos puedan estudiar. 

\end{enumerate}

\bibliography{../bibliografia/bibTex}{}
\bibliographystyle{plain}
